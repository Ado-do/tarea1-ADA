%!TeX program = xelatex
%!TeX encoding = UTF-8
\documentclass{style}

\begin{document}
%%%%%%%%%%%%%%%%% RENOMBRE %%%%%%%%%%%%%%%%%
\graphicspath{ {./img/} }
\renewcommand{\contentsname}{Índice de contenido}
\renewcommand{\listfigurename}{Índice de figuras}
\renewcommand{\listtablename}{Índice de Tablas}
\renewcommand{\tablename}{Tabla}
\renewcommand{\figurename}{Imagen}
\renewcommand*{\lstlistingname}{Código}


%%%%%%%%%%%%%%%%% ENCABEZADO %%%%%%%%%%%%%%%%%
% Si se quiere eliminar, se tiene que quitar tambien las configuraciones del style.cls
% \fancyhf{} % Para quitar el molesto encabezado con el nombre de la seccion actual !!!!!!!!!!!!!!!!
\fancyhead[L]{ % Encabezado a la izquierda
  \begin{picture}(0,0) \put(-20,16){\includegraphics[width=15mm]{logo-udec1.pdf}} \end{picture}
  \put(24,45){\textcolor{gray}{\scriptsize{\begin{tabular}{l}
    UNIVERSIDAD DE CONCEPCIÓN\\
    FACULTAD DE INGENIERÍA\\
    DEPARTAMENTO DE INGENIERÍA INFORMÁTICA Y CIENCIA DE LA COMPUTACIÓN
  \end{tabular}}}}
}


%%%%%%%%%%%%%%%%% PORTADA %%%%%%%%%%%%%%%%%
% Opcion 1: Importar un pdf tamaño carta de EEUU
%\includepdf[pages=-]{img/Portada.pdf} % cambiar portada por pdf
% Opcion 2: Escribir una portada completa
\begin{titlepage}
\includegraphics[width=1.6cm]{img/logo-udec1.pdf}
\vspace*{-2.2cm} % Esta linea va con la anterior


\begin{tabular}{l}
\hspace{2cm} UNIVERSIDAD DE CONCEPCIÓN\\
\hspace{2cm} FACULTAD DE INGENIERÍA\\
\hspace{2cm} DEPARTAMENTO DE INGENIERÍA INFORMÁTICA Y CIENCIA DE LA COMPUTACIÓN\\
\end{tabular}

\begin{center}
\vspace{3cm}
{\scshape\Huge \textbf{Tarea } \par}
\rule{80mm}{0.1mm}
%\vspace{1cm}

{\itshape\LARGE Informatica \par}
\vfill
% {\Large \textbf{Grupo XX} \par}
\vspace{0.5cm}
{\large \textbf{Alumno:}\\\normalsize Alonso Bustos \par} 


\medskip
\textbf{Fecha de entrega:} -- de ----, ----
\vspace{3cm}
\end{center}
\end{titlepage}
\newpage
% Opcion 3: Sin portada - Se recomienda quitar indices


%%%%%%%%%%%%%%%%% INDICES %%%%%%%%%%%%%%%%%
\setstretch{1}
\vspace*{\fill}
    \tableofcontents % No pude quitarle la negrita :c
    % \listoftables
    % \listoffigures
    %\lstlistoflistings % Para hacer indice de codigo
\vspace*{\fill}


%%%%%%%%%%%%%%%%% ESPACIADO %%%%%%%%%%%%%%%%%
\setstretch{1.15} 

%%%%%%%%%%%%%%%%% CONTENIDO %%%%%%%%%%%%%%%%%
% \newpage

\subfile{content/content.tex}

\subfile{content/tutorial} % Eliminar en caso de no requerir

%%%%%%%%%%%%%%%%% BIBLIOGRAFIA %%%%%%%%%%%%%%%%%
% \newpage
% Hay 2 formas de agregar bibliografia:
% 1. Agregar una bibliografia en un archivo .bib (Es super automatico)
% (Al agregar citas como \citet{einstein}, \cite{latexcompanion,knuthwebsite} se agregaran automaticamente)

% This document is an example of \texttt{natbib} package using in bibliography 
% management. Three items are cited: \textit{The \LaTeX\ Companion} book \cite{latexcompanion}, the Einstein journal paper \citet{einstein}, and the 
% Donald Knuth's website \cite{knuthwebsite}. The \LaTeX\ related items are
% \cite{latexcompanion,knuthwebsite}. 

\bibliographystyle{apacite}
\bibliography{content/sample.bib} % Requiere crear un archivo .bib

% 2. Agregar una bibliografia en un archivo .tex 
% (Es manual, pero comodo para los que no conoce el bibtex)
% \subfile{content/bibliografia}

\end{document}
