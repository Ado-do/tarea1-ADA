\documentclass[main.tex]{subfiles}
\begin{document}

\section{Dividir para Vencer (Constanza)}

Diseñe o busque en Internet (cite su fuente) un algoritmo \code{divide\_and\_conquer} usando
dividir para vencer para encontrar tal distancia en tiempo en $\smallo{n^2}$. Realice un análisis
de correctitud del algoritmo. Defina y demuestre su complejidad computacional.


\subsection{Diseño}

Se escoge el algoritmo explicado en el libro Introduction to Algorithms de Cormen et al.
(Sección 33.4), el cual utiliza el paradigma de Dividir para Vencer para resolver de forma eficiente
el problema del par de puntos más cercanos. El algoritmo que fue implementado para las
siguientes secciones se basa en:

\begin{description}
	\item[Previo a la recursíon:]
	      Se crean dos arreglos, $X$ e $Y$, que contienen todos los puntos del conjunto original.
	      El arreglo $X$ se ordena de forma creciente según la coordenada $x$, mientras que el
	      arreglo $Y$ se ordena de forma creciente según la coordenada $y$.

	\item[Dividir para Conquistar:]
	      Se divide en tres secciones:
	      \begin{enumerate}
		      \item \textbf{Caso base:}
		            Al comenzar se revisa si los puntos existentes en el array a trabajar son
		            menores o iguales a 3, de ser así se devuelve el resultado de
		            utilizar \code{brute\_force}.
		      \item \textbf{Dividir:}
		            \begin{enumerate}
			            \item Se calcula el punto medio del arreglo y se divide en 2 arreglos
			                  $X_i$ y $X_d$ que corresponden a la mitad de la izquierda y de la
			                  derecha respectivamente.

			            \item De manera similar se crean los arreglos $Y_i$ e $Y_d$, que guardan
			                  los puntos ordenados de manera creciente según la coordenada $y$ del
			                  lado izquierdo y derecho.

			            \item Se aplica \code{divide\_and\_conquer} a cada lado: en $X_i$ y $X_d$.

			            \item Esto da dos distancias mínimas: $\delta_i$ y $\delta_d$, que se
			                  comparan y escoge el mínimo entre las dos $\delta$.
		            \end{enumerate}
		      \item \textbf{Combinar:}
		            \begin{enumerate}
			            \item Se forma una franja vertical de ancho $2\delta$ centrada en la línea divisoria.
			            \item Se crea un arreglo $Y'$ que guarda aquellos puntos que existen a una distancia menor
			                  a $\delta$ de la linea divisoria, ordenados de manera creciente según la coordenada $y$.
			            \item Se comparan cada uno de los puntos del nuevo arreglo entre a lo más
			                  6 o 7 puntos siguientes del mismo. Esta propiedad geométrica se
			                  traduce a: comparar cada punto del arreglo con aquellos puntos con
			                  los que su distancia euclidiana no sobrepase la distancia mínima
			                  actual $\delta$.
			            \item Si se encuentra una distancia menor a $\delta$ dentro de esta franja,
			                  se actualiza el resultado.
		            \end{enumerate}
	      \end{enumerate}
\end{description}

\subsection{Correctitud}

Se aborda el análisis de la correctitud separando esta misma en parcial y total.

\subsubsection{\textbf{Correctitud Parcial}}
Se basa en que cada paso cumple su propósito:

\begin{itemize}
	\item  En el caso base, si hay 3 o menos puntos, el algoritmo recurre al método por fuerza
	      bruta, el cual es correcto por definición.
	\item En la fase de división, los puntos se separan correctamente en dos subconjuntos disjuntos
	      ($V_i$, $X_d$) y sus equivalentes ordenados por $y$), asegurando que se evalúe la
	      distancia mínima en cada mitad.
	\item En la combinación, se considera correctamente la franja central, y se comparan sólo
	      los  puntos relevantes mediante una propiedad geométrica fundamentada: en un
	      rectángulo de dimensiones $\delta \times 2\delta$ no puede haber más de 8 puntos,
	      separados al menos por $\delta$, lo que justifica comparar solo con los 6 o 7 puntos
	      siguientes en $Y'$.
\end{itemize}

\subsubsection{\textbf{Correctitud Total}}
Se garantiza porque:

\begin{itemize}
	\item La recursión siempre se aproxima al caso base, disminuyendo el tamaño del problema en cada llamada.

	\item En cada nivel, se obtiene una solución válida (correctitud parcial), y se combina correctamente
	      con la solución del subproblema opuesto.

	\item Al combinar los resultados locales de cada subproblema y la franja, se garantiza que el algoritmo
	      retorna globalmente la distancia mínima.
\end{itemize}

\subsection{Complejidad Computacional}
Defina y demuestre su complejidad computacional.
El algoritmo divide recursivamente un conjunto de $n$ puntos en dos subconjuntos de tamaño $n/2$,
y en cada nivel realiza trabajo adicional lineal: construcción de arreglos auxiliares ordenados
($Y_i$, $Y_d$ e $Y'$)  y evaluación de pares en la franja central. Gracias al ordenamiento inicial
de los puntos en orden creciente según su coordenada $x$ e $y$, estas operaciones pueden implementarse en tiempo \bigo{n} por nivel.

El tiempo de ejecución se puede modelar con la siguiente recurrencia:
\[
	T(n) = 2T(n/2) + \bigo{n}
\]
Esta recurrencia puede resolverse mediante el Teorema Maestro, que aplica a recurrencias de la forma:
\[
	T(n) = aT(n/b) + f(n)
\]
Donde:
\begin{itemize}
	\item $a$: Número de subproblemas
	\item $b$: Factor por el cual se divide el tamaño del problema.
	\item $f(n)$: Costo del trabajo fuera de las llamadas recursivas (división y combinación).
\end{itemize}

Se tiene que $a = 2$ y $b = 2$, entonces:
\[
	\log_b{a} = \log_2{2} = 1
\]
y dado que:
\[
	f(n) = \bigo{n} = \bigo{n^{\log_b{a}}}
\]
estamos en el caso del Teorema Maestro, en donde:
\[
	f(n) = \bigo{n^{\log_b{a}}}
	\rarrow
	T(n) = \bigo{n^{\log_b{a} \log{n}}} = \bigo{n \log{n}}
\]
Por lo tanto, la complejidad temporal del algoritmo es:
\[
	T(n) = \bigo{n \log{n}}
\]
\end{document}
