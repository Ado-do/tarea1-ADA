\documentclass[main.tex]{subfiles}
\begin{document}

\section{Fuerza Bruta (Alonso y Gabriela)}

Implemente un algoritmo \code{brute force} calculando las n(n − 1) distancias de manera a
seleccionar tal distancia mínima. Realice un análisis de correctitud del algoritmo. Defina y
demuestre su complejidad computacional.


\subsection{Implementación}

Este es el extracto de la implementación en código cpp (\code{brute\_force.cpp}) del algoritmo brute
force que calcula las n(n - 1) distancias en un conjunto de puntos:
\vspace{1em}
\lstinputlisting[style=cpp, firstline=9, lastline=32]{../../src/brute_force.cpp}
\vspace{1em}

\subsection{Correctitud}

\begin{description}
	\item[Objetivo del algoritmo:] ~\\
	      El objetivo principal del algoritmo es encontrar la distancia euclidiana mínima entre todos
	      los pares de puntos contenidos en un vector de estructuras \code{Point2D}.

	\item[Invariante del ciclo:] ~\\
	      \textbf{Definición de la invariante:} Al inicio de cada iteración del ciclo externo (índice
	      $i$), la variable \code{min\_dist} contiene la distancia mínima encontrada entre todos los
	      pares de puntos ya revisados, es decir, aquellos donde $0 \leq a < b \leq i + k$, siendo $k$
	      el número de iteraciones completas del ciclo interno (índice $j$).

	\item[Verificación de la invariante:] ~\\
	      Verificamos en cada momento del loop:

	      \begin{itemize}
		      \item \textbf{Antes del ciclo} \\
		            La variable \code{min\_dist} se inicializa con un valor suficientemente grande (la
		            distancia entre los puntos (0,0) y (100,100). En este punto, aún no se ha realizado
		            ninguna comparación, por lo que la invariante se cumple trivialmente.

		      \item \textbf{Durante el ciclo} \\
		            En cada iteración del ciclo doble, se compara la distancia entre el punto $i$ y
		            todos los puntos $j$ tales que $j > i$. Si la distancia calculada es menor que
		            \code{min\_dist}, esta se actualiza con el nuevo valor. En caso contrario,
		            \code{min\_dist} permanece inalterada. De este modo, al final de cada iteración,
		            \code{min\_dist} sigue conteniendo la distancia mínima entre todos los pares
		            evaluados hasta ese momento, cumpliendo así con la invariante.

		      \item \textbf{Después del ciclo} \\
		            Al finalizar la ejecución de ambos ciclos, se ha recorrido todo el conjunto de pares
		            posibles $(i, j)$ donde $i < j$, garantizando que todos han sido evaluados una sola
		            vez. Como resultado, la variable \code{min\_dist} contiene efectivamente la
		            distancia mínima global entre todos los pares de puntos del vector. Por lo tanto, se
		            cumple también la condición de salida del algoritmo.
	      \end{itemize}

	\item[Conclusión:] ~\\
	      Dado que la invariante se cumple inicialmente, se mantiene durante cada iteración, y al
	      finalizar garantiza la postcondición esperada, se concluye que el algoritmo es
	      \textbf{correcto} con respecto al problema que pretende resolver.
\end{description}


\subsection{Complejidad Computacional}

\begin{itemize}
	\item \textbf{Teórica}: \bigo{n^2} por el doble bucle anidado
	\item \textbf{Empírica}: Ver sección de análisis experimental
\end{itemize}

\end{document}
